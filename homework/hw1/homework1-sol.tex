\documentclass[a4paper, 11pt]{article}
\topmargin=-5mm
\evensidemargin=0cm
\oddsidemargin=0cm
\textwidth=16cm
\textheight=22cm
\addtolength{\headheight}{1.6pt}

\usepackage{amsmath}
\usepackage{enumitem}
\usepackage{listings}
\lstset{basicstyle=\ttfamily,language=ML}
\newcommand{\sml}[1]{\text{\texttt{#1}}}

\begin{document}
\title{FDD3023 ITP Homework 1}
\author{Karl Norrman, knorrman@kth.se}
\maketitle
%

This is typeset in latex to make the proofs a little easier to read.
%
The list datatype \hbox{is defined as}
%
\begin{lstlisting}
datatype 'a kList = kNil | kCons of 'a  * 'a kList
\end{lstlisting}
%

%------------------------------------------------------------------------ sect
\section{\texttt{kAppend}}
Prove that $\forall \mathtt{l}.\enskip$ \texttt{kAppend l kNil = l}.
%
The definition of \sml{kAppend} is
%
\vspace{10pt}
\begin{lstlisting}
fun kAppend kNil xs = xs
  | kAppend (kCons (x, xs)) ys = kCons (x, kAppend xs ys)
\end{lstlisting}
\vspace{10pt}
%
Proof by structural induction on second argument \sml{xs}.
%

\vspace{10pt}
\noindent Base case \sml{xs = kNil}:
\begin{enumerate}
\item \sml{kAppend kNil kNil = kNil} \hfill [by \sml{kAppend}]
\item qed -- base case
\end{enumerate}
%

\noindent Induction step. Assume ind-hypothesis: \sml{kAppend xs kNil = xs}.
%
Must show that $\forall$ \sml{x} \sml{xs}. \enskip
\sml{kAppend (kCons (x, xs)) kNil = kCons (x, xs)}.
%
\begin{enumerate}[resume]
    \item \sml{kAppend (kCons (x, xs)) kNil = kCons (x, (kAppend xs kNil))}
        \hfill  [by \sml{kAppend}]
    \item \sml{kCons (x, (kAppend xs kNil)) = kCons (x, xs)}
        \hfill [by ind-hypothesis]
    \item qed -- ind-step
\end{enumerate}
%

\section{\texttt{kAppend} and \texttt{kLength} interaction}
Prove that $\forall$ \sml{l1 l2}. \sml{length (append l1 l2) =
length l1 + length l2}.
%
Definitions of \sml{kAppend} and \sml{kLength} is as follows:
\vspace{10pt}
\begin{lstlisting}
fun kAppend kNil xs = xs
  | kAppend (kCons (x, xs)) ys = kCons (x, kAppend xs ys)

fun kLength kNil = 0
  | kLength (kCons (x, xs)) = 1 + kLength xs
\end{lstlisting}
\vspace{10pt}

\noindent Proof by structural induction on first argument \sml{l1}.
%

\vspace{10pt}
\noindent Base case \sml{l1 = kNil}:
\begin{enumerate}
    \item \sml{kLength (kAppend kNil l2) = kLength l2} \hfill [by \sml{kAppend}]
    \item \sml{kLength l2 = 0 + kLength l2} \hfill [by arithmetic]
    \item \sml{0 + kLength l2 = kLength kNil + kLength l2} \hfill [by \sml{kLength}]
    \item qed -- base case
\end{enumerate}

\noindent Induction step. Assume ind-hypothesis:
\sml{kLength (kAppend xs l2) = kLength xs + kLength l2}.
Must show that
\sml{kLength (kAppend (kCons (x, xs)) l2) = kLength (kCons (x, xs)) + kLength l2}.

\begin{enumerate}[resume]
    \item \sml{kLength (kAppend (kCons(x, xs)) l2) = kLength kCons(x, (kAppend
        (xs, l2)))} \hbox{[by \sml{kAppend}]}
    \item \sml{kLength kCons(x, (kAppend (xs, l2))) = 1 + kLength (kAppend (xs, l2))}
        \hbox{[by kLength]}
    \item \sml{1 + kLength (kAppend (xs, l2)) = 1 + kLength xs + kLength l2}
        \hbox{[by ind-hypothesis]}
    \item \sml{1 + kLength xs + kLength l2 = kLength (kCons (x, xs)) + kLength l2}
        \hbox{[by kLength]}
    \item qed -- ind-step
\end{enumerate}

\end{document}
